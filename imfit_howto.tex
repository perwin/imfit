\documentclass[10pt]{article}
\usepackage[]{mathpazo}

\setlength{\oddsidemargin}{1.0cm}
\setlength{\textwidth}{13.5cm}

\newcommand{\imfit}{\texttt{imfit}}
\newcommand{\Imfit}{\texttt{Imfit}}



\begin{document}

% Definition of title page:
\title{
  Notes for Using ``Imfit''
}
\author{
  Peter Erwin
}
\date{\today}  % optional

\maketitle

\section{What Is It?}

\Imfit{} is a program for fitting astronomical images --- more
specifically, for fitting images of galaxies, though it can in principle
be used for fitting other sources. The user specifies a set of one or
more 2D functions (e.g., elliptical exponential, elliptical S\'ersic,
circular Gaussian) which will be added together in order to generate a
model image; this model image will then be matched to the input image by
adjusting the 2D function parameters via nonlinear minimization of the total
$\chi^2$.

The 2D functions can be grouped into arbitrary sets sharing a common $(x,y)$
position on the image plane; this allows galaxies with off-center components
or multiple galaxies to be fit simultaneously. Parameters for the inividual
functions can be held fixed or restricted to user-specified ranges. The
model image can (optionally) be convolved with a Point Spread
Function (PSF) image to better match the input image; the PSF image can
be any square, centered image the user supplies (e.g., an analytic 2D Gaussian
or Moffat, a \textit{Hubble Space Telescope} PSF generated by the TinyTime
program [REF], or an actual stellar image).

A key characteristic of \imfit{} is a modular, object-oriented design
that allows relatively easy addition of new, user-specified 2D image
functions. This is accomplished by writing C++ code for a new
image-function class (this can be done by copying and modifying an
existing pair of \texttt{.h/.cpp} files for one of the pre-supplied
image functions), modifying one additional file to include references to
the new function, and re-compiling the program. Notes are [WILL BE] provided
to guide the interested user in adding new functions; in most cases, a
basic working knowledge of C should suffice.

Additional auxiliary programs built from the same codebase exist for
generating artificial galaxy images (using the same input/output
parameter-file format as \imfit{}).

\Imfit{} is an open-source project; the source code is freely available
under the GNU Public License (GPL).


\bigskip

\textbf{System Requirements:} \Imfit{} has been built and tested on
Intel-based MacOS X and Linux (Ubuntu) systems. It uses standard C++ and
should work on any Unix-style system with a modern C++ compiler and the
Standard Template Library (e.g., GCC v4 or higher). It relies on two
external, open-source libraries: version 3 of the CFITSIO
library\footnote{http://heasarc.nasa.gov/fitsio/} for FITS image I/O and
version 3 of the FFTW (Fastest Fourier Transform in the West)
library\footnote{http://www.fftw.org/} for PSF convolution.

MORE CREDITS -- e.g., Craig Markwardt's lmfit code; Differential Evolution
code.  [Mersenne Twister code??]



\section{Getting/Installing \Imfit{}}

\subsection{Pre-Compiled Binaries}

Pre-built binaries for Intel-based MacOS X and Linux systems are available XXX.


\subsection{Compiling from Source: Outline}

\begin{enumerate}
\item Install CFITSIO

\item Install FFTW

\item (Optional) Install GNU Scientific Library (GSL) --- this is only necessary
if you wish to use 2D image functions that rely on GSL; currently, the only
such function is the EdgeOnDisk (\texttt{func\_edge-on-disk.cpp}) component.

\item Install SCons

\item Build \imfit{}

\end{enumerate}

\subsection{Building \Imfit{} from Source}

Assuming that CFITSIO and FFTW (and optionally GSL) are already installed
on your system, XXX

\Imfit{} uses SCons for the build process; SCons is a Python-based build system
that is somewhat easier to use and more flexible than the traditional \texttt{make}
system. SCons can be downloaded from XXX


\section{Existing Image Functions}

\Imfit{} comes with the following set of 2D image functions, each of
which can be used as many times as desired. (As mentioned above, \imfit{}
is designed so that constructing and using new functions is a relatively
simple process.) Note that elliptical functions can always be made circular
by setting the ``ellpiticity'' parameter to 0.0 and specifying that it be
held fixed.

\begin{itemize}
\item FlatSky --- a uniform sky background.
\item Gaussian --- an elliptical 2D Gaussian function.
\item Moffat --- an elliptical 2D Moffat function.
\item Exponential --- an elliptical 2D exponential function.
\item Exponential\_GenEllipse --- an elliptical 2D exponential function using
generalized ellipses (``boxy'' to ``disky'' shapes).
\item S\'ersic --- an elliptical 2D S\'ersic function.
\item Sersic\_GenEllipse --- an elliptical 2D S\'ersic function using
generalized ellipses (``boxy'' to ``disky'' shapes).
\item FlatExponential --- similar to Exponential, but with an inner radial zone
of constant surface brightness.
\item BrokenExponential --- similar to Exponential, but with \textit{two}
exponential radial zones (with different scalelengths) bounded by a transition region
of variable sharpness.
\item GaussianRing2Side --- an elliptical ring with a radial profile
consisting of an asymmetric Gaussian (different values of $\sigma$ for
($r < R_{\rm ring}$ and $r > R_{\rm ring}$).
\item EdgeOnDisk --- the analytical form for a perfectly edge-on exponential
disk, using the Bessel-function solution of van der Kruit \& Searle (1981) for 
the radial profile and the generalized sech function of van der Kruit (1988) 
for the vertical profile. Note that this function requires that the GNU
Scientific Library (GSL) be installed; if the GSL is not installed, \imfit{}
will be compiled without the this function.
\item EdgeOnRing --- a simplistic model for an edge-on ring, using an
asymmetric Gaussian for the radial profile and a symmetric Gaussian (with
different $\sigma$) for the vertical profile.

%Exponential, Exponential_GenEllipse, Sersic, Sersic_GenEllipse, Gaussian, 
%FlatExponential, BrokenExponential, BrokenExponential2D, EdgeOnDisk, Moffat, 
%FlatSky, EdgeOnDiskN4762, EdgeOnDiskN4762v2, EdgeOnRing, GaussianRing2Side.

\end{itemize}

Note that a list of the currently available functions can always be obtained
by running \imfit{} with the ``\texttt{--list-functions}'' option:
\begin{quote}
  \texttt{\$ IMFIT --list-functions}
\end{quote}
and the complete list of function parameters for each function can always be
obtained by running \imfit{} with the ``\texttt{--list-parameters}'' option:
\begin{quote}
  \texttt{\$ IMFIT --list-parematers}
\end{quote}




\section{Using \Imfit{}}

Basic use of CMELT from the command line looks like this:
\begin{quote}
  \texttt{\$ CMELT }  [options] ~~ \textit{parameter-file} ~~
  [\textit{output-catalog}]
\end{quote}
where \textit{parameter-file} is the filename of the parameter file
which specifies the catalogs to combine and the structure of the
output catalog (see Section~\ref{sec:paramfile}), and
\textit{output-catalog} is an optional name for the combined catalog
(if you don't supply this name, then the default name is
\texttt{catalogmelt\_output.cat}).  By default --- i.e., without any
options specified --- the program will read a list of input catalogs
from \textit{parameter-file}, do a match using ``match-all'' mode (see
Section~\ref{sec:modes}) with a maximum match distance of 0.7 arc
seconds, and then write the combined output catalog (using columns
specified in \textit{parameter-file}) to disk.

There are various command-line options which can be used to specify
things like the maximum matching distance, minimum number of input
catalogs for a valid souce match, alternate coordinates for matching
sources, etc.; these are explained later.

The default mode is ``match-all,'' which simply means that only
sources with matches in \textit{all} input catalogs are combined and
written to the output catalog.  The default maximum match distance is
0.7 arc seconds; the $x$ and $y$ coordinates are assumed to be
spherical coordinates in decimal degrees, with column names of
\texttt{ALPHA\_SKY} and \texttt{DELTA\_SKY} in the input catalogs
(these are the standard SExtractor names).

\textit{A simple example}: Let's say you've written a parameter file
(see below) called, in this case, \texttt{combine\_params.dat},
containing the names of the input catalogs and the column
specifications for the output catalog.  The sources in each input
catalog have positions in Right Ascension and Declination in decimal
degrees, with column labels ``ALPHA\_SKY'' and ``DELTA\_SKY,''
respectively.  You decide that the maximum acceptable distance for a
match between catalogs is 0.5 arc seconds.  Now you type
\begin{quote}
  \texttt{\$ CMELT --dist=0.5 combine\_params.dat new.cat}
\end{quote}
and (after some time) the resulting catalog will be saved in the 
file \texttt{new.cat}.

Various bits of intermediate information are printed out as CMELT
runs, letting you know what stage of the process it's on.  If there
are a lot of matches \textit{and} a lot of output columns requested in
the parameter file, then the final stage may take a while: the program
will appear to hang after printing ``\texttt{Generating data text
lines for catalog...}'', but don't panic.  For scenarios with several
thousand matched sources and more than 100 output columns, this is
currently the slowest part of the process.

Typing \texttt{CMELT -h} will give you a short list of command-line
options and a very brief description; this can be useful once you have
some idea of what it does\ldots.


\subsection{Catalog Format}

Input catalogs to CMELT must be in a format similar to that
generated by SExtractor.  Specifically, they must be ASCII files with
a header (lines beginning with ``\#'') and the data must be in
whitespace-separated columns, with one line per source (there is no
real upper limit to the length of lines).  The header can have any
amount of extraneous information (timestamps, etc.), but \textit{must}
have a SExtractor-style column listing, one line describing each data
column:
\begin{quote}
  \texttt{\#~nnn~NAME~[extra information]}
\end{quote}
where $nnn$ is the column number (FORTRAN-format: 1 = first column, 2
= second column, etc.)  and NAME is a descriptive name for that
column.  There \textit{must} be one or more spaces surrounding the
column number.  The listing can have extra information after the name
(separated from NAME by whitespace), which is ignored.

Catalogs must also have, at a minimum, $x$ and $y$ coordinates for
each source, \textit{as numbers} --- no ``hh:mm:ss'' forms.  If the
coordinates are spherical (e.g., RA and Dec), then they should be in
\textit{decimal degrees}.

Here is an example:
\begin{quote}
  \texttt{\# The exceedingly simple source catalog for filter X}\\
  \texttt{\#   1 NUMBER~~~~~~~~~~Running object number}\\
  \texttt{\#   2 ALPHA\_SKY~~~~~~~Right ascension of barycenter (native)          [deg]}\\
  \texttt{\#   3 DELTA\_SKY~~~~~~~Declination of barycenter (native)              [deg]}\\
  \texttt{\#   4 MAGNITUDE}\\
  \texttt{    1    213.6606300   52.1853960    23.414}\\
  \texttt{    2    213.6815992   52.1781032    19.077}\\
\end{quote}

Finally, input catalogs should be sorted by the $x$ coordinate, in 
ascending order.



\section{The Parameter File}\label{sec:paramfile}

\texttt{Catalogmelt} always requires a parameter file, which for
simplicity I'll refer to as \texttt{combine\_params.dat} from this
point on.  (You can call it anything you like; I find it convenient to
name them \texttt{combine\_params\_}\textit{foo}\texttt{.dat}, with
\textit{foo} being a shorthand description of some kind.)  The
parameter file contains a list of input catalogs (file names),
followed by a blank line and then a list of output columns ---
basically, a list of desired data columns for the output (combined)
catalog, and the input-catalog sources for each column.  A trivial
example, assuming three catalogs derived from \textit{HST}
observations:

\begin{quote}
  \texttt{\# List of catalogs to combine, and short-hand id for each}\\
  \texttt{/home/erwin/F450W\_catalog.cat  : B}\\
  \texttt{/home/erwin/F555W\_catalog.cat  : V}\\
  \texttt{/home/erwin/F814W\_catalog.cat  : I}\\
  
  \texttt{\# Desired columns in combined catalog, source for each,}\\
  \texttt{\# and possible "no-source" text:}\\
  \texttt{ALPHA\_SKY           : B:4}\\
  \texttt{DELTA\_SKY           : B:5}\\
  \texttt{B\_ID                : B:1}\\
  \texttt{B\_MAG               : B:10}\\
  \texttt{V\_ID                : V:1}\\
  \texttt{V\_MAG               : V:10}\\
  \texttt{I\_ID                : I:1}\\
  \texttt{I\_MAG               : I:10}\\
\end{quote}

So in this case we want to combine three input catalogs (the first
block of non-comment lines), which we refer to as B, V, and I; the
output catalog will have eight columns (the second block), beginning
with RA and Declination, and followed by three pairs of ID's and
magnitudes.

The second block specifies the output columns, and where they come
from.  We want the first two columns in the output catalog to be
position (RA and Dec), and in this case we know that columns 4 and 5
in the input catalogs contain this data.  The third column is some
form of source ID (perhaps just a number) from column 1 of the input B
catalog.  Then we have some kind of B magnitude, from column 10 of the
B catalog.  And so forth.


XXX more text here

\subsection{Formatting Rules for Output Column Specifications}

The basic format is: desired name for the output column, followed by a
space-delimited colon, followed by one or more input-column specifiers

\begin{quote}
  \texttt{{\sl COLUMN\_NAME}  ~:~   }\textit{input-column specifier}
\end{quote}

Note that the colon \textit{must} be surrounded by one or more spaces
(or tabs) on each side.

The input-column specifier consist of one or more catalog-id + 
column-number pairs, separated by colons (here, \textit{no} space 
around the colons!).  Alternate catalog--column-number pairs can be 
listed, separated by forward slashes; they are for cases where an 
output source may not have a match in one or more catalogs.  Some 
examples:

\begin{quote}
  \texttt{NAME1  ~:~   B:57} \\
  \texttt{NAME2  ~:~   B:14~ /~ V:14~ /~ I:28}
\end{quote}
In this case, data for the output column ``NAME1'' is taken from
column 57 of the B catalog.  For NAME2, we have three options.  The
first choice is column 14 of the B catalog, and this is always used
\textit{unless} an output source had no match in the B catalog, in
which case column 14 of the V catalog is used.  If there was no V
match either, then column 28 of the I catalog is used.


\subsubsection{Defaults for Missing Matches}

Sometimes, there is no logical alternative if a source doesn't have a
match in a given input catalog.  E.g., an X-ray source has no match
in the $V$ catalog, for whatever reason.  But since you want one of
the output columns to be V\_MAG, \textit{something} has to go there.

The default value which is put in the output catalog is ``99.0000'', 
the same as SExtractor uses for bad or otherwise missing values.  You 
can change this for a given output column by adding a missing-value 
specifier, like so:
\begin{quote}
  \texttt{{\sl COLUMN\_NAME}  ~:~   
  }\textit{input-column-specifier}\texttt{~:~} 
  \textit{missing-value-specifier}
\end{quote}
The missing-value specifier is simply a string which will be written 
verbatim to the output catalog.  If you want a string with spaces in 
it, you must place it inside double quotation marks (otherwise, only 
the first part of your string will be used).

Some examples:
\begin{quote}
  \texttt{NAME1  ~:~   B:57} \\
  \texttt{NAME2  ~:~   B:14~ /~ V:14 ~ : ~ -1} \\
  \texttt{NAME3  ~:~   B:34 ~ : ~ "no value"}
\end{quote}
For NAME1, we want values from column 57 of the B catalog; when no B
match is found, the default of ``99.000'' is written to the output
catalog.  For NAME2, we want column 14 from the B catalog, or else
column 14 from the V catalog (if no B match is found).  If no B
\textit{or} V match is found, then ``-1'' is written to the output
catalog.  The final line says that the phrase ``no value'' should be
written instead of column 34 from the B catalog, when no B match is
found.


\subsubsection{Special Cases}

There is a special shorthand if you know that you want \textit{all} 
columns from a given input catalog used, in their original order and 
with their original names:
\begin{quote}
  \texttt{\$ALL\$  ~:~ } \textit{catalog-id}
\end{quote}
For example:
\begin{quote}
  \texttt{\$ALL\$  ~:~ B}
\end{quote}
This is basically a wild-card substitution: at this point in the
formatting of the output catalog, we want to use each and every column
from catalog B, with their original names.  You can do this for more
than one input catalog, of course --- but if the input catalogs have
columns with the same name, it will be difficult to tell them apart in
the output catalog\ldots.




\section{Matching Sources Between Catalogs}

\subsection{The Basic Idea}

CMELT looks for matches between sources in the input catalogs, using 
their positions and a maximum allowable distance value.  The basic 
algorithm is this:

\begin{enumerate}
  
  \item Take the first catalog in the list from the parameter file
  (Section~\ref{sec:paramfile}).  For each source in that catalog, do
  the following:
  
  \begin{enumerate}
    
    \item Search for matches in the next (``target'') catalog in the list by 
    comparing $x$ and $y$ positions.  If any sources in the target 
    catalog are closer than the maximum allowed distance, save the 
    \textit{closest} such source as a ``match'' and mark that target 
    source as matched.
    
    \item Repeat for all the other catalogs in the list.
  \end{enumerate}
  
  \item If using ``match-all'' or ``first'' modes, skip to Step 4.
  Otherwise, go on to the next catalog in the list.  For each source
  in this catalog which is not already marked as having been matched
  in a previous round of Step 1, repeat Step 1.
  
  \item Repeat until the number of catalogs left in the list is $< 
  N - 1$, where $N$ is set by the \texttt{--nmatch} option.
  
  \item Go through the set of matches, identifying and discarding 
  duplicate matches.
  
  \item Sort the remaining matched sources in ascending order of the 
  $x$-coordinate.
  
  \item Generate output data lines for the output catalog, using the 
  output-column specifications from the parameter file.  Use default 
  data values for cases with no match from a given catalog.
  
  \item Save the resulting output catalog.
\end{enumerate}




\subsubsection{Matching by Distance}

The fundamental way CMELT finds a possible match between sources in
two different catalogs is by calculating the distance between them,
using their $x$ and $y$ coordinates.  By default, $x$ and $y$ are
assumed to be Right Ascension and Declination (in decimal degrees),
and so the distance calculation includes a $\cos \delta$
spherical-coordinate correction.  If you \textit{don't} want the
spherical-coordinate correction --- i.e., if your coodinates are
Cartesian (pixel values, say) --- then you can use the
\texttt{--cartesian} command-line option.

By design, only the \textit{closest} sources are matched.  If source
104 from catalog B is within the maximum distance of three sources in
catalog V (sources 56, 57, and 75), then only the closest source from
catalog V is recorded as the match.  (This is different from how the
\textsc{iraf} task \texttt{tmatch} behaves: \texttt{tmatch} will save
all three matches (B-104 + V-56, B-104 + V-57, B-104 + V-75) to the
output catalog, along with a separate file listing all such multiple
matches and the corresponding distances.  It is then up to you to
decide what to do with such multiple matches.)

The default maximum allowed distance is 0.7 arc seconds.  This can be
changed with the \texttt{--dist=<D>} command-line options, where $D$
is a distance in arc seconds, or in absolute units if
\texttt{--cartesian} is also specified).



\subsection{Match Modes: Different Ways of Combining Catalogs}
\label{sec:modes}

This section describes the different ways CMELT can generate an 
output catalog, based on how many matches between catalog you require 
for a ``valid'' source.

The default is ``match-all'' mode, which simply means a source must be 
matched in \textit{all} input catalogs for it to be written to the 
output catalog.

A looser specification is with the \texttt{--nmatch=<N>} option, 
which sets the \textit{minimum} number of matches for a valid output 
source.  For example, if you have six catalogs listed in 
\texttt{combine\_params.dat}, then
\begin{quote}
  \texttt{\$ CMELT~--nmatch=4~combine\_params.dat}
\end{quote}
will save sources which match in four, five, or all six of the 
catalogs.  (Obviously, \texttt{--nmatch=6} is the same as match-all 
mode in this case.)  This method is useful if you want to include 
sources not detected in certain filters, e.g., if they are extremely 
red or blue.

Finally, you can also request that the \textit{first} catalog in the
parameter file's catalog list be a template for the output catalog:
the output catalog will consist of each and every source in that
particular input catalog.  This is done with the  \texttt{--first} option:
\begin{quote}
  \texttt{\$ CMELT~--first~combine\_params.dat}
\end{quote}
In this case, the point of matching sources between the catalogs is to
add information from the other catalogs to the first catalog.  E.g.,
if the first catalog is an X-ray catalog, and the others are optical
and near-IR catalogs, then the output catalog will be the X-ray
sources from the input catalog, with whatever additional information
comes from matches to the other catalogs (fluxes in other bands, more
accurate positions, morphological parameters, etc.).

There is also the option \texttt{--firstfilt=<catalog-id>}, which does
the same thing, except that the catalog specified by
\textit{catalog-id} is bumped up to the front of the list and used as 
the template, instead of whatever catalog is listed first in the 
parameter file.


\section{Miscellaneous Options and Alternatives}

\subsection{Timestamps}

Output catalogs automatically have a ``timestamp'' prepended to the 
header which looks like this:
\begin{quote}
\# Generated by catalogmelt on Wed Nov 10 14:51:05 2004 by user erwin (Peter Erwin) \\
\# with the following command-line invocation: \\
\#      ./catalogmelt.py combine\_params\_all.dat test.cat \\
\end{quote}

This can be turned off with the \texttt{--notimestamp} option.


\subsection{Debugging Output}

Various levels of debugging output\ldots

Can save match distances, too\ldots

XXX


\subsection{Options and Their Defaults}

\begin{enumerate}
  \item \textbf{maximum match distance:} Maximum allowed distance between 
  sources in different catalogs for sources to still be counted as 
  matched.  Default = 0.7 arc seconds (or 0.7 absolute units if 
  \texttt{--cartesian} is used).
  \begin{itemize}
    \item \texttt{--dist=<D>} --- change maximum allowed distance to 
    $D$.
  \end{itemize}
  
  \item \textbf{match mode:} See Section~\ref{sec:modes}.  Default is 
  ``match-all''; other modes are:
  \begin{itemize}
    \item \texttt{--nmatch=<N>} --- include sources with matches in 
    $N$ or more catalogs
    \item \texttt{--first} --- use the first input catalog as a 
    template for the output catalog.
    \item \texttt{--firstfilt=<catalog-id>} --- use the input catalog
    whose ID is \textit{catalog-id} as a template for the output
    catalog (e.g., \texttt{--firstfilt=B}).  This has the same effect
    as editing the parameter file to put that catalog first in the 
    list and then using \texttt{--first}, but is easier to do!
  \end{itemize}
  
  \item \textbf{Coordinates:} Default $x$ and $y$ coordinates are RA
  and Dec in decimal degrees, with column names \texttt{ALPHA\_SKY}
  and \texttt{DELTA\_SKY} in the input catalogs.
  \begin{itemize}
    \item \texttt{--cartesian} --- treat $x$ and $y$ coordinates as 
    Cartesian rather than spherical: no $\cos y$ correction will be 
    applied to distance calculations, and the match-distance is 
    assumed to be in absolute units rather than arc seconds.
    \item \texttt{--xcol=<XCOL>} --- $x$-coordinate column name 
    (default = \texttt{ALPHA\_SKY}) or number
    \item \texttt{--xcol=<XCOL>} --- $y$-coordinate column name 
    (default = \texttt{DELTA\_SKY}) or number
  \end{itemize}
  
  
\end{enumerate}


\section{Caveats and Limitations}

CMELT is optimized to work on catalogs covering small or moderate
areas of the sky.  If you have one or more all-sky catalogs, or
catalogs which wrap around $x = 0$ (i.e., RA $= 0$ for most
astronomical catalogs), then the assumptions behind the clever bits
which speed up the matching will be violated, and things won't work
right.  This is also the case if one or more of the catalogs are not
sorted in the $x$ coordinate.

In the future, there may be a ``brute-force'' option to help in cases 
like this\ldots. XXX

In certain cases, the exact matches found and saved to the 
output catalog can depend on the order in which the input catalogs 
are searched (i.e., the order listed in the parameter file).  At the 
moment, I'm not aware of a general solution to this; it seems to be an 
inherent uncertainty in the matching process.  The best workaround 
would seem to be putting the catalog with the most accurate 
coordinates first.  [XXX possibly insert an illustration of the 
problem here XXX]





\end{document}
